Electrons originating from non-prompt heavy-flavor decays are usually embedded within a jet and are surrounded by all the additional decay products. To separate these from prompt electrons, an isolation criteria is applied to each electron to suppress these background electrons.

Isolation is quantified through nearby tracks of charged particles and calorimeter energy deposits. The track-based isolation variable, $\pt^{varcone20}$, sums the \pt{} of all nearby tracks contained within a cone of $\Delta R = 0.2$ excluding those associated to the electron object. Calorimeter-based isolation uses a variable $\et^{cone20}$, which is built similarly to the track-based variable except the \et{} within a cone around the center of the electron object is summed together. One key difference is corrections are applied to remove contributions from the electron's own energy deposits, leakage, and pileup.

Two isolation WPs, Loose and Tight, are used throughout the $VH$ analysis and are supported by the electron CP group. The Loose WP requires $\et^{cone20}/\pt < 0.2$ for calorimeter isolation and $\pt^{varcone20}/\pt < 0.15$ for track isolation. The Tight WP has stricter requirements, requiring $\et^{cone20}/\pt < 0.06$ for calorimeter isolation, and $\pt^{varcone20}/\pt < 0.06$ for track-based isolation. 
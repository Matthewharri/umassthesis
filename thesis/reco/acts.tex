% A Common Tracking Software (ACTS)~\cite{Salzburger_A_Common_Tracking_2021, Ai:2021ghi} is a tracking software that began its development in 2016 with a small team within the ATLAS collaboration. Since then, ACTS has grown into an international collaboration with approximately 80 contributors, a majority of which are from the ATLAS collaboration. It is based on the track reconstruction algorithms developed and used within ATLAS, but is being developed to remove the tracking algorithms from the ATLAS software framework, Athena~\cite{athen_citation}, and develop a community driven track reconstruction software. ACTS provides the algorithms needed for track reconstruction via an open-source software kit hosted on Github~\cite{github_citation}, and presents them in a generic, framework, and experimental-agnostic way.

The A Common Tracking Software (ACTS) project~\cite{Salzburger_A_Common_Tracking_2021, Ai:2021ghi} began its development in 2016 with a small team within the ATLAS collaboration. Since then, it has grown into an international effort comprising approximately 80 contributors, a majority of whom are affiliated with ATLAS\@. ACTS builds upon track reconstruction algorithms originally developed for ATLAS, with the goal of decoupling them from the ATLAS software framework, Athena~\cite{atlas_collaboration_2021_4772550}, and providing a standalone, community driven tracking software package.

ACTS offers a set of algorithms for track reconstruction such as Kalman filter, Hough transform, GS, and AMVF\@. These are implemented in an open-source software toolkit that is publicly available on GitHub. The software is written in C\texttt{++}20~\cite{iso_cpp20}, a language widely used within the particle physics community due to its compatibility with existing frameworks and its high execution speed. Additionally, it is designed with minimal external dependencies and only relies on two third-party libraries, \texttt{Eigen}~\cite{eigenweb} and \texttt{Boost}~\cite{boost}. The build and dependency management systems are handled by \texttt{CMake}~\cite{cmake}.

Allowing other experiments to adopt ACTS for their track reconstruction was a key design goal. To achieve this, ACTS has been written to be intentionally generic, allowing it to be framework-independent and experiment-agnostic. Several experiments are currently investigating or actively using ACTS, including: Belle II~\cite{Belle-II:2010dht}, CEPC~\cite{CEPCStudyGroup:2023quu}, sPHENIX~\cite{Osborn:2021zlr}, PANDA~\cite{SMYRSKI201285}, FASER~\cite{FASER:2022hcn}, EIC~\cite{osti_1765663}, and is currently deployed for ATLAS ID track reconstruction in Run 3~\cite{Mlinarevic:2024ntj}. 

Current muon track reconstruction in ATLAS is based on the Athena implementation of tracking algorithms. However, the existing codebase is complex, difficult to maintain, and brittle, making it suboptimal for meeting the performance and scalability requirements of Run 4. As a result, muon track reconstruction is migrating to the ACTS for Run 4, due to its modernity, maintainablity, and performitivity.



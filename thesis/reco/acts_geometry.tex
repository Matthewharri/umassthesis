% A dedicated reconstruction geometry is necessary for the definition of surfaces, description of detector material, and for track extrapolation and track fitting integration. Commonly, high energy physics (HEP) experiments use a simplified version of the detector geometry for track finding and track fitting for a balance between computational performance and reconstructive accuracy. By assuming a Gaussian or multi-Gaussian noise to account for multiple scattering or energy loss, the detector material detector description can be simplified to a model with an averaged material description. High-quality track reconstructions depends on an accurate description of material in amount and position due to the affects that can be had from material interactions. 

A dedicated reconstruction (tracking) geometry is necessary for defining detector surfaces, describing material distributions, and enabling track extrapolation and fitting. Often, high energy physics experiments use a simplified detector geometry for track finding and fitting to strike a balance between computational performance and reconstruction accuracy. Material effects such as multiple scattering and energy loss are computationally expensive, and thus are typically modeled using Gaussian or multi-Gaussian noise assumptions, allowing the material description to be reduced to an averaged representation.~\cite{Salzburger:2007rba} Despite this simplification, accurate modeling of material distribution in amount and location, is crucial, as even small inaccuracies can significantly impact track quality due to interactions with detector material.

ACTS has undergone multiple iterations of its tracking geometry model, culminating in a third and final version. These successive models are commonly referred to as Gen 1, Gen 2, and Gen 3. The Gen 1 geometry was the initial implementation and closely followed the tracking geometry structure used in Athena. The Gen 2 model, often called the layer-less geometry, introduced a new paradigm that moved away from explicit layer definitions in favor of a more flexible, volume-based structure. The Gen 3 geometry, also known as the adaptive geometry, represents a hybrid approach that incorporates the strengths of both Gen 1 and Gen 2, resulting in a more robust, flexible, and performant geometry model.

The first geometry model is based upon \texttt{Surface} and \texttt{Volume} objects. The \texttt{Surface} and \texttt{Volume} objects, are purely geometrical constructs and are extended with \texttt{Layer} and \texttt{TrackingVolume} objects. The \texttt{Layer} object is an extension of \texttt{Surface} and can hold arrays of sensitive surfaces and their corresponding material description. The \texttt{TrackingVolume} object extends the \texttt{Volume} object and can hold arrays of contained layers, arrays of contained volumes, and volume based material descriptions. During geometry construction, any unoccupied regions within a \texttt{TrackingVolume} that contains an array of layers are filled with \texttt{NavigationLayer} objects, which ensure a fully static geometry in which every point in space is covered by a \texttt{Layer} object. The \texttt{TrackingGeometry} object is built using all \texttt{TrackingVolume} objects for the geometry, creating a fully connective geometry setup. The \texttt{TrackingVolume} objects have boundary surfaces which act as portals from one volume to the next along a particle trajectory. The navigation state is done in a global view.

The second geometry model introduces a different approach by eliminating the concept of layers and introducing three core components: \texttt{Portal}, \texttt{DetectorVolume}, and \texttt{Detector}. The \texttt{Portal} class is a composite of \texttt{Surface} objects and represents a geometric boundary with additional metadata describing connections to neighboring volumes. Each portal contains volume link information that is used during navigation to determine the next volume to enter based on the intersection point and normal vector at the portal surface. The \texttt{DetectorVolume} class serves a role analogous to the \texttt{TrackingVolume} in Gen 1, and is constructed using a \texttt{DetectorVolumeFactory}, which creates the portals and establishes the volume links. In contrast to Gen 1, navigation is delegated to the volumes, therefore each \texttt{DetectorVolume} requires a \texttt{InternalNavigationDelegate} which describes how the navigation inside the volume should be performed. At a minimum, it must provide access to the portals for exiting the volume, but can also provide internal sensitive surfaces if they exist. The \texttt{Detector} object acts as the top-level geometry container, analogous to the \texttt{TrackingGeometry} from Gen 1, and must include at least one \texttt{DetectorVolume}, as well as a \texttt{DetectorVolumeFinder}, which resolves spatial points to the appropriate \texttt{DetectorVolume}.

The third geometry model introduces a hybrid approach that combines aspects of the previous two models. In this version, the \texttt{TrackingVolume} is extended to include boundary portals and an internal navigation policy to manage navigation within the volume. The geometry construction has been entirely rewritten, and is now based upon a hierarchical tree model, with the \texttt{Blueprint} object at its root. The \texttt{Blueprint} object defines the overall layout and can be recursively expanded using nodes that represent subsystems, tracking layers, or volumes. Construction proceeds in three phases: build, connect, and finalize. In the build phase, the geometry is instantiated as a tree of containers, where only the leaf nodes represent actual geometry elements, and each node's final dimensions are computed based on a specified attachment strategy, with sizing performed radially or along the $z$-axis depending on orientation. The connect phase adds portals at volume boundaries using the finalized sizes and merges or fuses portal from different volumes. In the finalize phase, portals are registered to their volumes, volumes to their containers, and navigation policies are established.

The muon geometry has been built using all three different geometries. The details of these are presented in the following section.
% The third geometry model represents a hybrid approach combining elements of the two previous models. In this version, the \texttt{TrackingVolume} has been extended to incorporate portals at its boundaries along with an internal navigation policy that governs how navigation is performed within the volume. Additionally, the geometry construction has been completely rewritten and is now based upon a tree structure, allowing for a more flexible representation. At the core of this construction is the \texttt{Blueprint} object, which defines the hierarchical layout of the geometry and is the top-level of the geometry. The \texttt{Blueprint} can be recursively branched using various types of nodes that can represent subsystems, tracking layers, or tracking volumes. The geometry construction is divided into three distinct phases: build, connect, and finalize. During the build phase, the geometry is instantiated as a tree of containers with only the bottom nodes representing the geometry objects, and the final dimensions of each node are computed. This step also requires the specification of an attachment strategy that governs how volumes are connected within the hierarchy. Volume sizing is performed along either the radial or axial direction, depending on the orientation of the structure. During the connect phase portals are built at the volume boundaries using the resized volumes. Additionally during this stage, other types of nodes could be inserted into the tree and would be run here. For example, there exists a \texttt{MaterialDesignatorBlueprintNode} that is configured to assign material to the final portal surfaces. The finalize step registsers portals to their respective volumes, register the volumes to their respective containers, and creates the navigation policies.

% The third geometry model is a combination of both where the \texttt{TrackingVolume} object has evolved and now contains portals at the boundaries and an internal navigation policy which dictates how internal navigation proceeds. The geometry construction has been rewritten from the ground up and is based upon a tree structure. The central concept for the construction comes from a \texttt{Blueprint} object which can be branched upon with different nodes. Nodes can consists of subsystems, tracking layers, or tracking volumes. The \texttt{Blueprint} 

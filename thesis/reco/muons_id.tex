After reconstruction, several identification WPs are defined to distinguish prompt and non-prompt muons. These WPs consist of several criteria, including the number of hits in different ID subdetectors and MS stations, the quality of the track fit, and the compatibility between ID and MS tracks. 

The identification WPs primarily target muons from light hadrons, which can be distinguished from prompt muons due to their low-quality muon tracks and characteristic kinks in their trajectory due to in-flight decays. In contrast, muons that originate from a heavy-flavor decay produce high-quality muons tracks but originate from a secondary vertex, allowing them to be distinguished from prompt muons.

The Muon CP group~\cite{ATLAS:MCPDocs} provides a set of three identification WPs, listed in order of increasing purity and decreasing efficiency are: Loose, Medium, and Tight. These WPs form a heirarchy where muons passing a tighter WP are guaranteed to pass the looser ones. Additional WPs targeting more extreme regions of phase space like high or low $\pt$ muons exists, but are beyond the scope of this thesis. 

The Loose WP is optimized for analyses with a high signal-to-background ratio, such as Higgs boson decays to four muons, where maintaining a high efficiency is more important than stringent background reject. It covers the full ID acceptance of $|\eta| < 2.5$ and includes CB and IO muons. In the $|\eta| < 0.1$ range, CT and ST muons are allowed due to a gap in MS coverage leading to a loss of CB efficiency. IO muons with $\pt < 7$ GeV and $|\eta| < 1.3$ with hits in only one precision station are accepted provided that they are also reconstructed as ST muons, to boost efficiency in low $\pt$ regions. Using this WP with simulated $t\bar{t}$ events, approximately 97\% of muons are reconstructed as CB or IO muons, while about 1.5\% are reconstructed as CT and ST muons in the region $|\eta| < 0.1$.~\cite{ATLAS:2020auj}

The Medium WP provides a good balance between efficiency, purity, and small systematic uncertainties, making it suitable for most analyses. It is a subset of the Loose WP covering the range $|\eta| < 2.5$, accepting only CB and IO muons. Hits must be present in at least two precision stations, except in the region $|\eta| < 0.1$ where only one is required. In Run 2, the acceptance was extended beyond the ID coverage to include ME and SiF muons in the range $2.5 < |\eta| < 2.7$, with hits in at least three precision stations. However, the additional coverage is not yet supported in Run 3. For prompt muons that pass this WP in $t\bar{t}$ events, more than 98\% are reconstructed as CB muons.

The Tight WP is designed for analyses where non-prompt muon backgrounds dominate. Muons passing this WP are a subset of the Medium and Loose WPs requiring the muon to have hits in at least two precision stations and a high-quality track fit. This WP is optimized to reject non-prompt muons and achieves a 50\% reduction in some regions of phase space when compared to the Medium WP, while maintaining only a 6\% efficiency loss for prompt muons. 

%%%%%%%%%% Can i use Run 2 results since Run 3 plots are not public -- Loose, Medium and Tight WP plots of efficiencies  vs eta and pt with tracks > 10 GeV in simulation

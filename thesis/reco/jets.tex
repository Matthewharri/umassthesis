Jets originate from the fragmentation and hadronization of quarks and gluons produced in the $pp$ interaction. During fragmentation, partons produced in the collision radiate additional partons, which then undergo hadronization, a process in which they combine, creating hadrons. In ATLAS, this results in energy deposits forming a cone shape around the original parton, which is identified as a jet. Since individual partons cannot be directly observed, jets serve as their physical proxy.

The fragmentation and hadronization processes lead to a wide variety of jet signatures. Some hadrons decay in-flight producing electrons, photon, neutrinos, or other hadrons. As a result, there is no single jet signature. To address this, ATLAS developed a particle flow (PFlow)~\cite{ATLAS:2017ghe} algorithm, which matches ID tracks to calorimeter energy deposits. This algorithm results in improved energy and mass resolution at low jet \pt{} and better pile-up stability compared to older jet reconstruction algorithms.~\cite{ATL-PHYS-PUB-2022-038}
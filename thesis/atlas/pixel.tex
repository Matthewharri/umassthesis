As mentioned, the ID is composed of three different sub-detectors. The innermost sub-detector is the pixel detector~\cite{pixel_detector_2008}~\cite{pixel_detector_ibl}. The pixel detector consists of four layers in the barrel region, and three forward disks on either side of the IP\@. When a charged particle passes through the pixel detector, it ionizes the silicon, creating electron-hole pairs along the path it takes. A strong bias voltage is applied across the sensor, so when the electron-hole pairs are created, they are pulled in opposite directions and drift towards electrodes, producing an electric signal. 
The original ATLAS pixel detector features 80 million, small rectangular pixels, with dimensions of 50 microns in the $\varphi$ direction and 400 microns in the $z$ or $r$ direction depending on barrel or end-cap region. These pixels are grouped into sensor modules with each module containing 46,080 pixels. Each module is mounted onto a mechanical and cooling support structure called a stave. Staves in the barrel region contain 13 modules, while the end-cap equivalent to staves, sectors, each contain 6 modules.

The Insertable B-Layer (IBL) is the innermost sub-detector of the ATLAS Pixel Detector system and was added between Run 1 and Run 2 to enhance tracking performance under the higher luminosity conditions expected in later runs. Positioned just 3.3 cm from the beam pipe, the IBL incorporates two different sensor technologies, planar sensors and 3D sensors, to maintain or improve the robustness and performance of ATLAS tracking given the higher luminosities expected during Run 2 and beyond. Unlike the original pixel detector, the IBL has different sized pixels, allowing for more granular measurements. Each pixel in the IBL is 50 microns in the $\varphi$ direction and 250 microns in the $z$ or $r$ direction, resulting in a 60\% smaller pixel. The IBL consists of 14 staves, with each stave containing 32 modules and each module containing 26,880 pixels. This results in an additional 12 million readout channels for the ID\@.

Following the IBL are the three original barrel pixel layers, often referred to as B-Layer 0, B-Layer 1, and B-Layer 2, located at 5.05 cm, 8.85 cm, and 12.25 cm from the beam pipe, respectively. These layers contain 22, 38 and 52 staves respectively resulting in 13.2, 22.8, and 31.2 million readout channels for a combined 67 million readout channels.

In addition to the barrel layers, the Pixel Detector includes three end-cap disks on each side of the interaction point. These disks are positioned at approximately 40.05 cm, 58.0 cm, and 65.0 cm from the IP, with each disk containing 8 sectors, resulting in approximately 13.3 million more readout channels.
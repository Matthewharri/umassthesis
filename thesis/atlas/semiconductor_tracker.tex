% Surrounding the pixel detector is the Semiconductor Tracker (SCT)~\cite{atlas_sct}, whose purpose, like the pixel detector, is to detect charged particles as they propagate away from the IP\@. Hits in this sub-detector are detected via the same mechanisms as the pixel detector\@. Different from the pixel detector, however, is that this sub-detector consists of 4088 modules of silicon-strip detectors. Each module consists of 770 strips, with the modules arranged in four concentric barrels around the pixel-detector. Additionally, there are two end-cap parts of the SCT each consisting of nine disks. The barrel layer of the SCT contains 528 silicon-strip detectors, and the end-cap regions contain 494 strip-detectors each. In the barrel region, modules are glued back to back to each other resulting in a maximum of 8 hits being registered in the barrel region, while the end-caps contain 9 single-sided modules, resulting in a maximum of 9 hits. 

%%%%%%%%%%%%%% Revision

Surrounding the pixel detector is the Semiconductor Tracker (SCT)~\cite{atlas_sct} which detectors charged particles via silicon ionization in a similar mechanism to the pixels. Rather than using silicon pixels, the SCT is composed of 4088 modules of silicon-strip detectors each with 770 strips. The SCT consists of four concentric layer of modules surrounding the pixels in the barrel region and nine disks in both end-cap regions. The barrel layer of the SCT contains 528 silicon-strip detectors while the end-caps contain 494 strip-detectors each. In the barrel region, modules are glued back to back resulting in a maximum of 8 deposits of energy, while the end-caps contain 9 single-sided modules resulting in a maximum of 9 energy deposits. 
ATLAS uses a right-handed coordinate system, with origin located at the interaction point (IP). The $z$-axis runs along the beam line, the positive $x$-axis points towards the center of the LHC, and the positive $y$-axis points upwards. The positive $z$-axis is referred to as `side A', with the negative side referred to as `side C'.  More conveniently, polar coordinates are used in the transverse plane to take advantage of the cylindrical symmetry of the detector.
The standard notation for polar coordinates is used:

\begin{equation}
    r = \sqrt{x^2 + y^2}
\end{equation}\label{eq:transverse_radius}

\begin{equation}
    \varphi = \arctan\left(\frac{y}{x}\right)
\end{equation}\label{eq:transverse_phi}

\noindent{}The angle between the beam axis and the particles trajectory is defined as the polar angle $\theta$:

\begin{equation}
    \theta = \arctan\left(\frac{r}{z}\right)
\end{equation}\label{eq:polar_angle}

\noindent{}In particle physics it is more common to use the pseudorapidity $\eta$, which represents the angle of a particle with respect to the beam axis, and is defined as:

\begin{equation}
    \eta = -\ln\left(\tan\left(\frac{\theta}{2}\right)\right)
\end{equation}\label{eq:pseudorapidity}

\noindent{}Thus the commonly used, and preferred, coordinate system is represented by (r, $\varphi$, $\eta$) where the distance between two objects can be defined by:

\begin{equation}
    \Delta R = \sqrt{{(\Delta \eta)}^2 + {(\Delta \varphi)}^2}
\end{equation}\label{eq:delta_R}

\noindent{}An illustration of the coordinate system can be seen in Figure~\ref{fig:atlas_coordinate_system}.

\begin{figure}
    \centering
    \includegraphics[width=0.9\textwidth]{figures/atlas/atlas_coordinate_system.jpg}
    \caption{Depicted is the traditional (x,y,z) coordinate system overlaid with the cylindrical coordinate system used by physicists. Taken from~\cite{atlas_coordinate_system}}\label{fig:atlas_coordinate_system}
\end{figure}

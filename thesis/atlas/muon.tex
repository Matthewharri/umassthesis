The outermost part of the ATLAS detector is the Muon Spectrometer (MS), designed to detect charged particles that escape the ID and calorimeter, and to measure their momentum in the pseudorapidity range of $\eta < 2.7$. Due to their relatively large mass, muons are less affected by bremsstrahlung radiation, and are typically the only particle to reach the MS\@. The MS was designed with two primary objectives: precise momentum measurements (resolution $\leq 50 ~\mu$m) and a fast muon trigger (15--25 ns). To achieve this, the MS uses five different gas-based detectors: The Monitored Drift Tube~\cite{atlas_mdt_paper} (MDT) chambers, the Resistive Plate Chambers~\cite{atlas_run3_rpc_paper} (RPC), the Thin Gap Chambers (TGC)~\cite{atlas_tgc_paper}, the Micromegas~\cite{atlas_mm_paper} (MM) and the Small-strip Thin Gap Chambers~\cite{atlas_stgc_paper} (sTGC). Refer to Figure~\ref{fig:atlas_detector} to see the layout of the MS\@.

To obtain precision momentum measurements, the MS uses three large superconducting magnets to bend muon trajectories. MDT's are primarily used for precision momentum measurements due to their high accuracy, predictable deformations, and simplicity of construction.

The precision-tracking chambers are complimented by a system of fast trigger chambers, which are capable of delivering tracking information with 15--25 ns after the passage of a particle. In the barrel region ($|\eta| < 1.05$) RPC's were selected for the triggering chambers, and in the end-caps ($1.05 < |\eta| < 2.4$) the TGC's were chosen. The timing resolution of the triggering detectors is finer than the LHC bunch spacing, allowing for reliable bunch crossing identification.

For Run 3, the sTGC and MM detectors were added to the MS and are known together as the New Small Wheel (NSW). These detectors compliment both the precision tracking detectors and fast triggering capabilities, by providing additional tracking and triggering information in the end-cap region.

The Standard Model (SM) of elementary particle physics is, to date, the best description that physicists have of fundamental interactions. This theory, which was developed throughout the
20th century, has been supported by numerous experimental observations and predicts key measurements with extreme precision. 

In June of 2012, the last missing piece of the standard model was finally observed at the Large Hadron Collider (LHC), the Higgs boson. In a joint announcement, both the ATLAS~\cite{ATLAS_Higgs_discovery} and CMS~\cite{CMS_Higgs_discovery} collaborations announced the discovery of a particle consistent with the Higgs boson at the approximate mass of
125 GeV. Since the discovery, many precision measurements of the Higgs boson have occurred between both collaborations, further confirming properties of the Higgs boson that are predicted by the Standard Model. Although the Standard Model has held up very well with all of its predictions, it has also been well established that the Standard Model is not a complete theory, with many key phenomenon still unexplained. 
For example, the Standard Model does not address dark matter or dark energy which together, make up a large part of the universe. Furthermore, gravity is not explained, offers no explanation for the matter-antimatter asymmetry, and suffers from the so-called `naturalness problem', which arises due to fine-tuning of the Higgs boson's mass to prevent large quantum corrections, a situation considered unnatural and indicative of missing physics.

Many extensions of the Standard Model, referred to as beyond the Standard Model (BSM), have been proposed to address its limitations mentioned previously, including the infamous, and most well known, supersymmetry~\cite{MARTIN_1998}, which introduces a new symmetry between bosons and fermions and offers solutions to some of the previously mentioned issues. However, after extensive searching, there has been no experimental evidence that supersymmetry exists.
In addition to direct BSM searches, model-independent approaches have also been developed that look for deviations from the Standard Model in precision measurements. One of these approaches is known as the Simplified Template Cross Sections~\cite{STXS_1_1} (STXS) framework, which has been developed centrally at the LHC via the LHC Higgs Working Group. This framework defines a set of fiducial regions in phase space which aims to maximize sensitivity to BSM effects while also minimizing dependence on theoretical predictions.

One such way to interpret potential deviations in a systematic way is to use Standard Model Effective Field Theory (SMEFT). This essentially extends the Standard Model lagrangian via perturbations and allows for the addition of higher order operators which are suppressed by higher energy scales. The strength of these operators is governed by parameters known as `Wilson coefficients'. By measuring cross sections in the STXS framework, and comparing to SMEFT predictions, constraints can be placed on the Wilson coefficients which allows us to probe BSM physics even without a direct signal.
An additional use of the STXS framework is that it allows for combination measurements with various decay channels, and even across experiments. These combinations provide increased statistical power and allow for more stringent testing of the Standard Model.

In this thesis, the inclusive, and several STXS cross sections, of Higgs boson production in association with a vector boson with the Higgs boson decay to $WW^{*}$ using 2022 and 2023 Run 3 ATLAS data will be presented. Chapter~\ref{ch:theory} will provide an overview of the Standard Model, and also a derivation of the Standard Model lagrangian. Afterwards, we will move on to talking specifically about the Higgs boson and its specific production and decay modes. Chapter~\ref{ch:atlas} will provide an overview of the LHC and ATLAS detector. Chapter~\ref{ch:data_mc} will discuss the datasets used in this analysis as well as the Monte Carlo simulations used. Chapter~\ref{ch:reco} will discuss event reconstruction with an emphasis on upgrade muon software work. Chapter~\ref{ch:nn} will discuss neural networks and illuminate the mysterious black box. Chapter~\ref{ch:vh_analysis} will discuss the VH analysis, and the different Higgs boson decay modes that were analyzed in this analysis. Chapter~\ref{ch:vh_inclusive_results} will discuss the results for the inclusive Higgs boson cross section measurement, and Chapter~\ref{ch:vh_stxs_results} will discuss cross section results in STXS bins.
Finally, we will conclude with a summary of the results and a discussion of future work in Chapter~\ref{ch:conclusions}.


% , while also discussing muon track reconstruction software for the future High-Luminosity LHC (HL-LHC)
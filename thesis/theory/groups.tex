In the derivation of the Standard Model Lagrangian, group theory plays an essential role. A group $G$ is defined as a set of elements with a binary operation $\ast$ that follow four axioms:
\begin{enumerate}
  \item G is closed under $\ast$, meaning $\forall a, b \in G, \; a \ast b \in G$  
  \item The operation $\ast$ is associative, meaning $\forall a, b, c \in G, \; (a \ast b) \ast c = a \ast (b \ast c)$.
  \item Each element has an identity, meaning $\exists I \in G$ so that $\forall a \in G, \; I \ast a = a \ast I = a$.
  \item Each element has an inverse, meaning $\forall a \in G, \; \exists b \in G$ so that $a \ast b = b \ast a = I$.
\end{enumerate}
Additionally, a group is said to be abelian if it's elements are commutative under $\ast$. 

In physics, symmetry groups are used to describe transformations that leave physical systems unchanged. The Standard Model is built on a special class of symmetry groups known as Lie groups, which is a continuous group whose elements can be parameterized by smooth transformations. The simplest example is $\mathrm{U}(1)$, the group of all $1 \times 1$ unitary matrices. This group is abelian and corresponds to phase rotations in the complex plane. In Section~\ref{sec:theory_qed} this group will play an important role in the formalism of quantum electrodynamics. 

Another important class of Lie groups are the special unitary groups, denoted by $\mathrm{SU}(N)$. These are a non-abelian group, consisting of all $N \times N$ unitary matrices with determinant 1. Only two such groups are relevant for the Standard Model: $\mathrm{SU}(3)$ (Section~\ref{sec:theory_qcd}) and $\mathrm{SU}(2)$ (Section~\ref{sec:theory_ew}).

These symmetry groups form the mathematical foundation for the structure of the Standard Model: $\mathrm{U}(1)$ for electromagnetism, $\mathrm{SU}(2)$ for the weak interaction, and $\mathrm{SU}(3)$ for the strong interaction. Together, they define the group $\mathrm{SU}(3)C \times \mathrm{SU}(2)L \times \mathrm{U}(1)Y$, which determines how elementary particles interact. The Standard Model Lagrangian is given by the following:
\begin{equation}
  \mathcal{L}_{\mathrm{SM}} = \mathcal{L}_{\mathrm{strong}} + \mathcal{L}_{\mathrm{EW}} + \mathcal{L}_{\mathrm{Yukawa}} + \mathcal{L}_{\mathrm{Higgs}}
  \label{eq:sm_lagrangian}
\end{equation}
which encapsulates these symmetries and will be explored in detail in the sections that follow.
